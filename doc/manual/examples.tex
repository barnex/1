\section{Examples}

\subsection{Standard problem 4}

\begin{verbatim}
# Micromagnetic standard problem 4
from mumax import *

# Material parameters:
msat(800e3)   
aexch(1.3e-11)     
alpha(0.02)

# geometry 
nx = 128
ny = 32
gridSize(nx, ny, 1)    
partSize(500e-9, 125e-9, 3e-9)

# initial magnetization
uniform(1, 1, 0)
alpha(2)

#relax:
alpha(1)
run(10E-9)
alpha(0.02)
save("m", "omf")

# run
autosave("m", "omf", 10e-12)
autosave("m", "png", 10e-12)
applyStatic('field', -24.6E-3, 4.3E-3, 0) # Apply a static external field
run(1e-9)
\end{verbatim}



\subsection{Geometry mask}

\begin{verbatim}

# material parameters:
msat(800e3)   
aexch(1.3e-11)     
alpha(1)


# geometry 
nx = 128
ny = 32
gridSize(nx, ny, 1)    
partSize(500e-9, 125e-9, 3e-9)
mask('shape.png') # While pixel: remove cell, black pixel: keep cell

# initial magnetization
setmRandom() # random magnetization
alpha(2)
run(10E-9)
save("m", "omf")

\end{verbatim}

\subsection{Custom-defined field}


\begin{verbatim}
# This example illustrates how to define an arbitary field
# B = f(t)
# current densities j(t) can be defined similarly.

from mumax import *
from math import sin, pi

# Nickel:
msat(490e3)   
aexch(9e-12)     
alpha(0.01)
k1(-5.7e3)
anisUniaxial(0, 0, 1)

# geometry 
nx = 128
ny = 128
gridSize(nx, ny, 1)    
partSize(300e-9, 300e-9, 50e-9)

# initial magnetization
uniform(1, 0, 0)

# custom-defined oscillating field along Y-axis
A = 1e-3    	# RF amplitude (T)
def myfield(t):
    f = 3000e6  # frequency = 3GHz
    by = A * sin( 2*pi * f * t)
    return 0, by, 0

# apply the custum defined field "myfield" for 2ns, sample its value every 10ps
# (linear interpolation between samples)
applyFunction('field', myfield, 2e-9, 10e-12) 

# current densities j(t) (in A/m^2) can be defined similarly:
applyFunction('j', myfield, 2e-9, 10e-12) 

# schedule output and run
autosave('table', 'ascii', 20e-12)
autosave('m', 'png', 50e-12)
run(2e-9) # run for 2ns

\end{verbatim}









