\section{Running}

\subsection{Launching the program}

To run \prog, type the following code in a terminal:

\shell{\prog \textit{arguments} \textit{input.in}}

Where \textit{\file{input.in}} is the name of the input file to run. It does not necessarily have to end with ``\file{.in}'' but we use this convention throughout this manual. The \textit{\cmd{arguments}} are optinal and described further on. 

\subsection{Input file}

A \prog input file specifies an entire simulation, including which output should be saved. Each file contains a series of \idx{commands} which are executed in the order they are provided. Let us start with a simple example:


\begin{verbatim}
# material
msat       	800E3 
aexch      	1.3E-11
alpha      	0.02
# geometry 
size       	1     32          128    
cellsize   	3E-9  3.90625E-9  3.90625E-9  # m
# initial magnetization
uniform		1 1 1
# run
autosave	table	ascii	10E-12
run          	1E-9
\end{verbatim}

\paragraph{Comments} All text after a hashmark (\#) is considered a \idx{comment} and is ignored by the simulation. In the above example, the statements \cmd{\# material}, \cmd{\# geometry}, \ldots are thus comments. They are only included for clarity and could be omitted.

\paragraph{Commands} All the other text in the input file is treated as a series of \idx{commands} that are executed in the order they are specified.

\subsection{Output}


Upon running an input file, an \idx{output directory} with a corresponding name but ending with ``\file{.out}'' will be created to store the simulation output. It also contains a file \idxfile{output.log} that keeps a log of all the output that appeared on the screen.  


\subsection{Program arguments}\label{arguments}

To obtain a list of program arguments, run \prog \cmd{-help}.
