\section{Installation}


\subsection{Operating system}

This program was devolped and tested on Ubuntu Linux version 10.04, and we expect it to run on most Linux distributions. It should in principle be possible to compile it on any Unix-like system, perhaps with minor modifications to the make scripts or source files. You are welcome to report issues or submit patches for other platforms.

\subsection{GPU Hardware}

To take advantage of the GPU acceleration, you need a \idx{CUDA capable} \textsc{NVIDIA} GPU. All recent \textsc{NVIDIA} GPUs should work, though it is highly recommended to use a ``Fermi''-architecture GPU with a \idx{compute capability} of at least 2.0 like the GeForce GTX series, Quadro 4000 or later, Tesla *2050 or later. For more info see \url{http://www.nvidia.com/object/cuda_gpus.html}

\subsection{GPU Drivers}

You will also need to install the proprietary \textsc{NVIDIA} graphics \idx{driver}. This driver may already be installed on your system. E.g., under Ubuntu, the ``Hardware drivers'' program may propose to install this driver. You should try to run a simulation on the GPU -- if that works, it means your driver is up to date.

Otherwise, you can install the driver located in \file{dep/cuda}. You have to install it when your X-server is not running: logout from your grahpical session, then use the ``\idx{Console login}'' of your login manager (or press Ctrl-Alt-F1) to obtain a text console. Log in and \cmd{cd} to \file{dep/cuda} the \prog root directory. Type\footnote{Or \cmd{sudo ./devdriver\_3.1\_linux\_32\_256.40.run} on a 32-bit system.}\footnote{Possibly a later version of this driver may be present}

\shell{sudo ./devdriver\_3.1\_linux\_64\_256.40.run}

and follow the instructions. Then reboot your computer. Note that you may have to re-install this driver after every kernel upgrade. To check your system, you can run the progam \idxcmd{nvidia-settings} which shows information about your GPU and driver.

It is possible that the required driver conflicts with Ubuntu's default version. If installing the driver fails, a possible workaround is to edit \file{/etc/modprobe.d/blacklist.conf} (as root\footnote{execute, e.g., \cmd{sudo gedit /etc/modprobe.d/blacklist.conf}}) and add these lines:
\small
\begin{verbatim}
blacklist vga16fb
blacklist nouveau
blacklist rivafb
blacklist nvidiafb
blacklist rivatv
\end{verbatim}
\normalsize

Then also remove all distribution-provided \textsc{NVIDIA} drivers with \cmd{sudo apt-get --purge remove nvidia-*}. Then follow the above instructions to re-install the driver provided by \textsc{NVIDIA}.

Not that you may have to run an X-server (i.e., have a graphical desktop installed) for the GPU driver to work correctly. So if you run ubuntu server edition, you may have to install ubuntu-desktop or kubuntu-desktop.

\subsection{Setup}

We recommend downloading a pre-compiled version of the program for either a 32-bit or 64-bit architecture. The 64-bit version is recommended if your system supports it. You can place the \prog directory anywhere on your hard drive. The only needed setup is running 

\shell{./setup\_64bit.sh}

(or \cmd{./setup\_32bit.sh} on a 32-bit system). This will create the \cmd{\prog} program in the \file{bin/} directory. Note that you have to re-run the setup 
script if you moved the \prog directory to a different location on your hard drive. You can now start the the program by typing \cmd{/path/to/\prog/bin/\prog} in a terminal, where you replace \file{/path/to/\prog} by the actual location where you put the \prog directory.

It is recommended to edit the (hidden) \file{.bashrc} file in your home directory and the following line: \cmd{export PATH=\$PATH:/path/to/\prog} where you replace \file{/path/to/\prog} by the actual location where you put the \prog directory. When you now open a new terminal, you will be able to start the program by just typing \prog.


