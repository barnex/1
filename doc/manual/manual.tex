\documentclass[a4paper, twoside]{article}

\usepackage{a4wide}
%\usepackage[Sonny]{fncychap}% Bjornstrup, Sonny, Lenny, Glenn, Conny, Rejne and Bjarne
\usepackage[final]{graphicx}
%\graphicspath{{fig/}}
\usepackage[english]{babel}
\usepackage{url}
\usepackage[small, bf, centerlast]{caption}
\usepackage{xspace}
\usepackage[latin1]{inputenc}      
\usepackage{textcomp}
\usepackage{amsmath}
\usepackage[final]{pdfpages}
%\usepackage[square]{natbib}
\bibliographystyle{unsrt}
%\bibpunct{[}{]}{,}{y}{,}{,} % maak y n
\usepackage{fancyhdr}
\pagestyle{fancy}
\fancyhf{}
% \fancyhf[HL]{\nouppercase{{\textsf{\arne}}}}        % Links in header het hoofdstuk,
% \fancyhf[HR]{\textsf\textbf{{\thepage}}}            % Rechts het paginanummer.
\usepackage{makeidx}
\makeindex
\usepackage{color}
\setcounter{MaxMatrixCols}{20}
\usepackage{listings}
\lstset{
language=C,                  % choose the language of the code
basicstyle=\sffamily\scriptsize,       % the size of the fonts that are used for the code
%backgroundcolor=\color{lightblue},  % choose the background color. You must add \usepackage{color}
showstringspaces=false,         % underline spaces within strings
showtabs=false,                 % show tabs within strings adding particular underscores
frame=none,                   % adds a frame around the code
tabsize=2,                      % sets default tabsize to 2 spaces
captionpos=b,                   % sets the caption-position to bottom
breaklines=true,                % sets automatic line breaking
breakatwhitespace=false         % sets if automatic breaks should only happen at whitespace
}


\newcommand{\prog}{\textsc{Simulate}\xspace}	% preliminary program name
\newcommand{\doctitle}{Simulations manual}

\usepackage[plainpages=false, colorlinks=true, citecolor=blue, linkcolor=blue, filecolor=blue, urlcolor=blue, bookmarks=true, pdftitle={\doctitle}, pdfauthor={Arne Vansteenkiste, Ben Van de Wiele}]{hyperref}
\usepackage[all]{hypcap} % link to image shows image, not caption.
\usepackage{backref} %biblio backreferences

% put a word in the text and index
\newcommand{\idx}[1]{\emph{#1}\index{#1}}
% used for terminal commands 
\newcommand{\cmd}[1]{\texttt{#1}}
\newcommand{\idxcmd}[1]{\texttt{#1}\index{#1}}
\newcommand{\file}[1]{\texttt{#1}\index{#1}}
% example code
\newcommand{\example}[1]{\fbox{\parbox{\textwidth}{\texttt{ #1 }}}}

\begin{document}

\hypersetup{breaklinks=true}
\setlength{\parindent}{0cm}

\title{\doctitle}
\author{Arne Vansteenkiste\\Ben Van de Wiele}
\maketitle


\tableofcontents

\section{Getting started}

\subsection{Requirements}

\subsubsection{Operating system}

This program was devolped and tested on Ubuntu Linux version 10.04, and we expect it to run on most Linux distributions. It should in principle be possible to compile it on any Unix-like system, perhaps with minor modifications to the make scripts or source files. You are welcome to report issues or submit patches for other platforms.

\subsubsection{Hardware and drivers}

To take advantage of the GPU acceleration, you need a CUDA capable \textsc{NVIDIA} GPU. All recent GPUs should be fine, though it is recommended to use a ``Fermi''-type GPU 

\subsection{Running}

\subsection{Re-compiling}





\section{Input files}

\newcommand{\command}[1]{\hyperref[#1]{\textbf{#1}}\index{#1}\label{#1}}

\mumax input files are written in Python. This way, one can build up a micromagnetic simulation suitable for each application by including loops, if-clauses, etc.  An introduction to the Python programming language can be found at \url{http://www.python.org/}.  Moreover, in order to set parameters, launch procedures, save output, etc. \mumax-specific commands are to be used.  Here is a simple example input file: \file{standardproblem4.py}

\begin{verbatim}
from mumax import *

# material
msat(800e3)
aexch(1.3e-11)
alpha(0.02)

# geometry 
gridsize(128, 32, 1)
partsize(500e-9, 125e-9, 3e-9)

# initial magnetization
uniform	(1, 1, 0)
relax(1e-5)

# run
autosave('table', 'ascii', 10E-12)
run(1e-9)
\end{verbatim}                                                                                                                                            

All text after a hashmark (\#) is considered a \idx{comment} and is ignored by the simulation.  They are only included for clarity and could be omitted.  All other text in the Python file is treated as a series of \idx{commands} that are executed in the order they are specified. In general, the order of the commands matters but should be easy to deduce.  E.g., you can not call \command{run} to start the time evolution when you have not first specified the material parameters, simulation size, etc\ldots. On the other hand, after having {run} the simulation for some time, you \emph{can} change the material parameters and call commands like \command{run} again. Either way, the program will tell you if it can not run a certain command yet because some parameters should be set first.

In what follows, we comment on the different \mumax-specific commands.

\subsection{Material parameters}

\begin{itemize}
 \item \textbf{msat}(\textit{arg}):\\
				Sets the saturation magnetization to the value specified in A/m.\\
				\textit{arg}: saturation magnetization in A/m.
 
 \item \textbf{setmsat}(\textit{i, j, k, scale}):\\
				Sets the saturation magnetization of the cell with index i,j,k to scale*msat\\
				\textit{i,j,k}: integer index of the cell. \textit{scale}: saturation magnetization scale factor.

 \item \textbf{aexch}(\textit{arg}):\\
				Sets the exchange constant to the value specified in J/m.\\
				\textit{arg}: exchange constant in J/m.

 \item \textbf{alpha}(\textit{arg}):\\
				Sets the damping coefficient to the specified value.\\
				\textit{arg}: damping coefficient.

 \item \textbf{setalpha}(\textit{i,j,k,scale}):\\
				Sets the damping coefficient of a single cell to alpha*scale.
				\textit{i,j,k}: integer index of the cell. \textit{scale}: alpha scale factor.

 \item \textbf{k1}(\textit{arg}):\\
				Sets the first order anisotropy constant K1 to the specified value in J/m$^3$.  Should be used in combination with anisuniaxial\\
%				Sets the first order anisotropy constant K1 to the specified value in J/m$^3$.  Should be used in combination with anisuniaxial or anis cubic.\\
				\textit{arg}: first order anisotropy constant in J/m$^3$.

 \item \textbf{k2}(\textit{arg}):\\
				Sets the second order anisotropy constant K2 to the specified value in J/m$^3$.  Only for cubic anisotropy.  Should be used in combination with \textrm{anisuncubic}\\
%				Sets the first order anisotropy constant K1 to the specified value in J/m$^3$.  Should be used in combination with anisuniaxial or anis cubic.\\
				\textit{arg}: second order anisotropy constant in J/m$^3$.

 \item \textbf{anisuniaxial}(\textit{arg1, arg2, arg3}):\\
				Defines the uniaxial anisotropy axis, normalization is done internally.\\
				\textit{arg1}: projection of anisotropy axis along the $x$-axis.\\
				\textit{arg2}: projection of anisotropy axis along the $y$-axis.\\
				\textit{arg3}: projection of anisotropy axis along the $z$-axis.

 \item \textbf{k1}(\textit{arg}):\\
				Sets the uniaxial anisotropy constant K1 to the specified value in J/m$^3$.\\
				\textit{arg}: first order uniaxial anisotropy constant in J/m$^3$.

\end{itemize}

\newcommand{\defcommand}[2][\space]{\textbf{#2}\index{#2}\label{#2} \textit{#1}}


\subsubsection*{Part size}
To define the magnet size, you must specify \emph{exactly two} of the three commands below.\\

\begin{tabular}{ll}
\defcommand[x y z]{partsize}  & Sets the size of the magnet, specified in m. \\
\defcommand[x y z]{cellsize}  & Sets the size of the finite difference cells, in m. \\
\defcommand[$N_x$ $N_y$ $N_z$]{gridsize} & Sets the number of finite difference cells.
\end{tabular}\\
\bigskip

\begin{itemize}
 \item After having set two of these values, the remaining one is calculated automatically. E.g. the part size and cell size automatically fix the number of cells.
 \item For performance reasons, the number of cells in each direction should preferentially be a power of two.
 \item For a 2D simulation, one can simply use $N_x \times N_y \times 1$ cells. In that case, optimized algorithms for a 2D geometry are used. Note that only the last dimension ($Z$) can be $1$ cell large, e.g., $1 \times N_y \times N_z$ is not a valid grid size.
\end{itemize}


\subsubsection*{Initial magnetization state}
\begin{tabular}{ll}
\defcommand[$m_x$ $m_y$ $m_z$]{uniform}  & Initializes the magnetization to a uniform state.\\
\defcommand[amplitude]{addnoise} & Adds random noise with the specified amplitude to a previously\\& defined magnetization state.
\end{tabular}




\subsubsection{Applied magnetic field}
\begin{tabular}{ll}
\defcommand[$B_x$ $B_y$ $B_z$]{staticfield}  & Applies a static magnetic field, specified in tesla.\\
\end{tabular}

\subsubsection*{Scheduled output}
\begin{tabular}{ll}
\defcommand[what format interval]{autosave}  & $what$ = m (magnetization) or table.\\
& $format$ = ascii, binary or png.\\
& $interval$ = save output every $interval$ seconds.
\end{tabular}


\subsection{Output}


Upon running an input file, an \idx{output directory} with a corresponding name but ending with ``\file{.out}'' will be created to store the simulation output. It also contains a file \idxfile{output.log} that keeps a log of all the output that appeared on the screen.  




\appendix


%\bibliography{biblio}

\printindex


\end{document}
