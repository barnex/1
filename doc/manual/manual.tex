\documentclass[a4paper, twoside]{article}
\usepackage{a4wide}
%\usepackage[Sonny]{fncychap}% Bjornstrup, Sonny, Lenny, Glenn, Conny, Rejne and Bjarne
\usepackage[final]{graphicx}
%\graphicspath{{fig/}}
\usepackage[english]{babel}
\usepackage{url}
\usepackage[small, bf, centerlast]{caption}
\usepackage{xspace}
\usepackage[latin1]{inputenc}      
\usepackage{textcomp}
\usepackage{amsmath}
\usepackage[final]{pdfpages}
%\usepackage[square]{natbib}
\bibliographystyle{unsrt}
%\bibpunct{[}{]}{,}{y}{,}{,} % maak y n
\usepackage{fancyhdr}
\pagestyle{fancy}
\fancyhf{}
% \fancyhf[HL]{\nouppercase{{\textsf{\arne}}}}        % Links in header het hoofdstuk,
% \fancyhf[HR]{\textsf\textbf{{\thepage}}}            % Rechts het paginanummer.
\usepackage{makeidx}
\makeindex
\usepackage{color}
\setcounter{MaxMatrixCols}{20}
\usepackage{listings}
\lstset{
language=C,                  % choose the language of the code
basicstyle=\sffamily\scriptsize,       % the size of the fonts that are used for the code
%backgroundcolor=\color{lightblue},  % choose the background color. You must add \usepackage{color}
showstringspaces=false,         % underline spaces within strings
showtabs=false,                 % show tabs within strings adding particular underscores
frame=none,                   % adds a frame around the code
tabsize=2,                      % sets default tabsize to 2 spaces
captionpos=b,                   % sets the caption-position to bottom
breaklines=true,                % sets automatic line breaking
breakatwhitespace=false         % sets if automatic breaks should only happen at whitespace
}


\newcommand{\prog}{\textsc{GSpin}}	% preliminary program name

\newcommand{\doctitle}{Simulations manual}

\usepackage[plainpages=false, colorlinks=true, citecolor=blue, linkcolor=blue, filecolor=blue, urlcolor=blue, bookmarks=true, pdftitle={\doctitle}, pdfauthor={Arne Vansteenkiste, Ben Van de Wiele}]{hyperref}%a4paper
\usepackage[all]{hypcap} % link to image shows image, not caption.
\usepackage{backref} %biblio backreferences

% put a word in the text and index
\newcommand{\idx}[1]{\emph{#1}\index{#1}}

\begin{document}

\hypersetup{breaklinks=true}
\setlength{\parindent}{0cm}

\title{\doctitle}
\author{Arne Vansteenkiste\\Ben Van de Wiele}
\maketitle


\tableofcontents

\section{Getting started}

\subsection{Requirements}

\subsubsection{Operating system}

This program was devolped and tested on Ubuntu Linux version 10.04, and we expect it to run on most Linux distributions. It should in principle be possible to compile it on any Unix-like system, perhaps with minor modifications to the make scripts or source files. You are welcome to report issues or submit patches for other platforms.

\subsubsection{Hardware and drivers}

To take advantage of the GPU acceleration, you need a CUDA capable \textsc{NVIDIA} GPU. All recent GPUs should be fine, though it is recommended to use a ``Fermi''-type GPU 

\subsection{Running}

\subsection{Re-compiling}






\appendix


%\bibliography{biblio}

\printindex


\end{document}
