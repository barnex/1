\section{What \mumax solves}

\subsection{Landau-Lifshitz-Gilbert formulation}

\mumax solves the Landau-Lifschitz equation:
\begin{equation}
\begin{split}
\frac{\partial\mathbf{M}(\mathbf{r}, t)}{\partial t} =
&- \frac{\gamma}{1+\alpha^2}\mathbf{M}(\mathbf{r}, t)\times \mathbf{H}_{eff}(\mathbf{r}, t)\\
&- \frac{\alpha \gamma}{M_s(1+\alpha^2)}\mathbf{M}(\mathbf{r}, t)\times\left(\mathbf{M}(\mathbf{r}, t)\times\mathbf{H}_{eff}(\mathbf{r}, t)\right).\label{LLequation}
\end{split}
\end{equation}

Note that for some reason the Gilbert factor $1/(1+\alpha^2)$ is \emph{not} included in OOMMF. So at high damping there can be discrepancies with OOMMF.

\subsection{Magnetostatic field}

\mumax calculates the magnetostatic field assuming constant magnetization in each cell:

\begin{equation}
\mathbf{H}_{ms}(\mathbf{r})
= -\frac{1}{4\pi}\int_V \nabla\nabla\frac{1}{|\mathbf{r}-\mathbf{r}'|}\cdot\mathbf{M}(\mathbf{r}')
\,\mathrm{d}\mathbf{r}'.
\label{Hms}
\end{equation}

This calculation is speed-up by exploiting its convolution structure and using FFTs. The magnetostatic kernel (magnetic field due to a single magnetized cell) is evaluated numerically. Currently, we integrate over the uniformly charged faces of the cell and evaluate the field at the center of the cells.

\section{Exchange}

\mumax uses a linear approximation for the exchange field:
\begin{equation}
\mathbf{H}_{exch} = \frac{2A}{\mu_0 M_s}\nabla^2\mathbf{m},\label{Hexch}
\end{equation}
The Laplacian is currently implemented as a 6-neighbor approximation:

\begin{equation}
\begin{split}
\mathbf{H}_{exch} = \frac{2A}{\mu_0 M_s} (&  \frac{m_x(1, 0, 0) -2m_x(0, 0 ,0) + m_x(-1, 0, 0)}{(\Delta_x)^2} \vec{e_x} +\\&  \frac{m_y(1, 0, 0) -2m_y(0, 0 ,0) + m_y(-1, 0, 0)}{(\Delta_y)^2} \vec{e_y} + \\& \frac{m_z(1, 0, 0) -2m_z(0, 0 ,0) + m_z(-1, 0, 0)}{(\Delta_z)^2} \vec{e_z} )
\end{split}
\end{equation}

\section{Anisotropy}

Uniaxial:
\begin{equation}
	\vec{H}_\mathrm{anis} = \frac{K_1}{\mu_0 M_s} (\vec{m} \cdot \vec{u}) \vec{u}
\end{equation}

This definition of $K_1$ corresponds to OOMMF. $K_1 > 0$: easy axis, $K_1 < 0$: hard axis.

\section{Spin-transfer torque}

\mumax incorporates the spin-transfer torque description developed by Berger \cite{Berger1996}, refined by Zhang and Li \cite{Zhang2004}
\begin{equation}
\begin{split}
\frac{\partial \mathbf{M}}{\partial t} = &-\frac{\gamma}{1+ \alpha^2}\mathbf{M}\times \mathbf{H}_{eff} \\
&- \frac{\alpha\gamma}{M_s(1+\alpha^2)}\mathbf{M}\times(\mathbf{M}\times \mathbf{H}_{eff})\\
&- \frac{b_j}{M_s^2(1+\alpha^2)}\mathbf{M}\times\left(\mathbf{M}\times (\mathbf{j}\cdot\nabla)\mathbf{M}\right)\\
&- \frac{b_j}{M_s(1+\alpha^2)}(\xi-\alpha) \mathbf{M}\times (\mathbf{j}\cdot\nabla)\mathbf{M}.\label{STT}
\end{split}
\end{equation}
Here, $\xi$ is the degree of non-adiabicity and $b_j$ is the coupling constant between the current density $\mathbf{j}$ and the magnetization
\begin{equation}
b_j = \frac{P \mu_B}{eM_s(1+\xi^2)},
\end{equation}
with $P$ the polarization of the current density,  $\mu_B$ the Bohr magneton and $e$ the electron charge.



